\documentclass[italian, 12pt]{article}
\usepackage[italian]{babel}
\usepackage[utf8]{inputenc}
\usepackage[letterpaper, left=1in, right=1in, bottom=0.75in, top=0.75in]{geometry}
\usepackage{amsmath}
\usepackage{subfiles}
\usepackage{lipsum}
\usepackage{csquotes}
\usepackage{amsfonts}
\usepackage[sfdefault]{plex-sans}
\usepackage{float}
\usepackage{pifont}
\usepackage{mathabx}
\usepackage[euler]{textgreek}
\usepackage{makecell}
\usepackage{tikz}
\usepackage{wrapfig}
\usepackage{siunitx}
\usepackage{amssymb} 
\usepackage{tabularx}
\usepackage{adjustbox}
\usepackage[document]{ragged2e}
\usepackage{floatflt}
\usepackage[hidelinks]{hyperref}
\usepackage{graphicx}
\usepackage{hyperref}
\setcounter{tocdepth}{4}
\usepackage{caption}
\usepackage{multicol}
\usepackage{tikz}
\setlength\parindent{0pt}
\captionsetup{font=footnotesize}
\usepackage{fancyhdr} 
\usepackage{graphicx}
\usepackage{capt-of}% 
\usepackage{booktabs}
\usepackage{varwidth}
\usepackage{colortbl}
\usepackage{xcolor}
\usepackage{listings}
%New colors defined below
\definecolor{codegreen}{rgb}{0,0.6,0}
\definecolor{codegray}{rgb}{0.5,0.5,0.5}
\definecolor{codepurple}{rgb}{0.58,0,0.82}
\definecolor{backcolour}{rgb}{0.95,0.95,0.92}
\definecolor{beigeCover}{rgb}{0.82, 0.77, 0.68}

%Code listing style named "mystyle"
\lstdefinestyle{mystyle}{
  backgroundcolor=\color{backcolour}, commentstyle=\color{codegreen},
  keywordstyle=\color{magenta},
  numberstyle=\tiny\color{codegray},
  stringstyle=\color{codepurple},
  basicstyle=\ttfamily\footnotesize,
  breakatwhitespace=false,         
  breaklines=true,                 
  captionpos=b,                    
  keepspaces=true,                 
  numbers=left,                    
  numbersep=5pt,                  
  showspaces=false,                
  showstringspaces=false,
  showtabs=false,                  
  tabsize=2
}

%"mystyle" code listing set
\lstset{style=mystyle}
\def\Title{Verbale Esterno 18/10/2024}
% \def\VerbaleIE{} da sistemare per farlo binario
\def\Author{SweeTen Team}
\def\Redattore1{Nicolas Fracaro}
\def\Verificatore1{Orlando Ferazzani}


\linespread{1.2}
\captionsetup[table]{labelformat=empty}

% -- TITOLO -- %
\newcommand{\customtitle}{\Title} % o ESTERNO

% -- PER LA FIRMA -- %
\newcommand{\signatureline}[1]{%
	 \par\vspace{0.5cm}
	\noindent\makebox[\linewidth][r]{\rule{0.2\textwidth}{0.5pt}\hspace{3cm}\makebox[0pt][r]{\vspace{3pt}\footnotesize #1}}%
}

% -- INTESTAZIONE -- %
\fancypagestyle{mystyle}{
	\fancyhf{} 
	\fancyhead[R]{\includegraphics[height=1cm]{images/logos/sweetenteam.png}} 
    \fancyhead[L]{\leftmark} 
   	\renewcommand{\headrulewidth}{1pt} 
  	\fancyhead[L]{\customtitle} 
	\renewcommand{\headsep}{1.3cm} 
	\fancyfoot[C]{\thepage} 
}

\begin{document}
\pagestyle{mystyle}
% Set background color for cover page
\pagecolor{beigeCover}

\include{"1 - Candidatura/lettera_di_presentazione/variables"}
\begin{titlepage}
	\begin{tikzpicture}[remember picture, overlay]
		\node[anchor=south east, opacity=0.5, yshift = -4cm, xshift= 2em] at (current page.south east) {\includegraphics[width=0.7\textwidth, trim=0cm 0cm 5cm 0cm, clip]{images/logos/Universita_Padova_transparent.png}}; 
		\node[anchor=north west, opacity=1, yshift = 8cm, xshift= 1.3cm, scale=1.6] at (current page.south west) {\includegraphics[width=4cm]{images/logos/sweetenteam.png}};
	\end{tikzpicture}
	
	\begin{minipage}[t]{0.47\textwidth}
		{\large\textbf{{\textsc{Destinatari}}}
			\vspace{3mm}
			\\ \large{\textsc{Prof. Tullio Vardanega}}
			\\ \large{\textsc{Prof. Riccardo Cardin}}
		}
	\end{minipage}
	\hfill
	\begin{minipage}[t]{0.47\textwidth}\raggedleft
		{\large\textbf{{\textsc{Redatto da: }}}
			\vspace{3mm}
			{\\\large{\textsc{ \Redattore1}\\}} % massimo due 
			
		}
		\vspace{8mm}
		
		{\large\textbf{{\textsc{Verificato da: }}}
			\vspace{3mm}
			{\\\large{\textsc{\Verificatore1}\\}} % massimo due 
			
		}
		\vspace{4mm}\vspace{4mm}
	\end{minipage}
	\vspace{4cm}
	\begin{center}
		\begin{flushright}
			{\fontsize{30pt}{52pt}\selectfont \textbf{\Title \\}}% \Date \\}} % o ESTERNO
		\end{flushright}
		\vspace{3cm}
	\end{center}
	\vspace{10 cm}
	{\small \textsc{\href{mailto: sweetenteam@gmail.com}{sweetenteam@gmail.com}}}
\pagestyle{mystyle}
\end{titlepage}

% Reset background color for the rest of the document
\pagecolor{white}


%-----------tabella versioni-----------%
\begin{table}[!h]
	\caption{Versioni}
	\begin{center}
		\begin{tabular}{ c c p{6.7cm} c c}
			\hline \\[-2ex]
			\textbf{Ver.} & \textbf{Data} & \textbf{Descrizione} & \textbf{Autore} & \textbf{Verificatore}  \\
			\\[-2ex] \hline \\[-1.5ex]
            2.0 & 10/11/2024 & Aggiunte specifiche su \textit{"Metodo rotazione ruoli"} in seguito ai feedback del professore & Nicolas Fracaro & Mouad Mahdi\\
            1.0 & 30/10/2024 & \textbf{Approvazione per Candidatura} & Nicolas Fracaro & Mouad Mahdi\\
			0.2 & 28/10/2024 &  Integrazione documento & Nicolas Fracaro & Mouad Mahdi\\
			0.1 & 28/10/2024 & Prima stesura del documento & Nicolas Fracaro & Mouad Mahdi\\
			\\[-1.5ex] \hline
		\end{tabular}
	\end{center}
\end{table}
%---------------------------------------%
\section*{Contributo del gruppo}
\noindent Tutti i membri del gruppo SweeTen Team (Valeri Mihail Belenkov, Davide Benedetti, Matteo Campagnaro, Orlando Ferazzani, Nicolas Fracaro, Mouad Mahdi, Andrea Santi) hanno contribuito equamente alla redazione del presente documento.
\newpage

\tableofcontents
\newpage


\section{Scopo del documento}
Il presente documento ha lo scopo di definire formalmente gli impegni del gruppo SweeTen Team per lo sviluppo del progetto BuddyBot proposto dall'azienda Azzurrodigitale. In particolare, vengono definiti:
\begin{itemize}
    \item La suddivisione dei ruoli e relative responsabilità
    \item Il preventivo dei costi
    \item La data di consegna prevista
    \item Gli impegni orari di ciascun membro del gruppo
\end{itemize}

\section{Impegni orari e suddivisione dei ruoli}
Ogni componente del gruppo si impegna a dedicare al progetto un totale di 91 ore produttive. I ruoli verranno ricoperti a rotazione da tutti i membri del gruppo, garantendo una distribuzione equa del carico di lavoro.



\subsection{Suddivisione ore e costi}
\begin{table}[h]
\centering
\begin{tabular}{lrrrr}
\toprule
\textbf{Ruolo} & \textbf{Costo orario (€)} & \textbf{Ore totali} & \textbf{Costo totale (€)} & \textbf{Ore per membro} \\
\midrule
Responsabile & 30,00 & 63 & 1.890,00 & 9 \\
Amministratore & 20,00 & 56 & 1.120,00 & 8 \\
Analista & 25,00 & 77 & 1.925,00 & 11 \\
Progettista & 25,00 & 119 & 2.975,00 & 17 \\
Programmatore & 15,00 & 168 & 2.520,00 & 24 \\
Verificatore & 15,00 & 154 & 2.310,00 & 22 \\
\midrule
\textbf{Totale} & & \textbf{637} & \textbf{12.740,00} & \textbf{91} \\
\bottomrule
\end{tabular}
\end{table}

\subsection{Ruoli e responsabilità}
Di seguito sono elencati i ruoli previsti con relative responsabilità:
\begin{itemize}
    \item \textbf{Responsabile:} Coordina il gruppo, gestisce le risorse e rappresenta il progetto verso l'esterno
    \item \textbf{Amministratore:} Gestisce l'ambiente IT di lavoro, gli strumenti e le procedure tecniche
    \item \textbf{Analista:} Studia il dominio del problema e definisce i requisiti
    \item \textbf{Progettista:} Definisce l'architettura e le scelte progettuali
    \item \textbf{Programmatore:} Implementa e mantiene il codice secondo le specifiche
    \item \textbf{Verificatore:} Assicura la qualità del prodotto attraverso verifiche e validazioni
\end{itemize}

\section{Partizione dei ruoli}
La distribuzione delle ore rappresenta una pianificazione orientativa che mira a coprire adeguatamente le principali attività, riservandosi la possibilità di adattamenti in corso d’opera, ferma restando l'invariabilità del monte ore complessivo e del relativo preventivo economico stabilito.

La struttura scelta cerca di bilanciare le responsabilità tecniche e gestionali, assicurando una gestione efficace delle risorse, uno sviluppo coerente con i requisiti progettuali e verifiche continue sulla qualità del prodotto finale. 

La distribuzione delle ore è stata definita seguendo i seguenti criteri:
\begin{itemize}
    \item \textbf{Focus sui ruoli tecnici}: Vista la natura del progetto, le ore sono state orientate principalmente verso ruoli come Programmatore e Progettista, fondamentali per una solida implementazione e architettura.
    \item \textbf{Qualità garantita dal Verificatore}: Il Verificatore ha un’allocazione rilevante per assicurare controlli continui, garantendo affidabilità e conformità del prodotto.
    \item \textbf{Equilibrio nella gestione}: I ruoli gestionali (Responsabile, Amministratore) sono distribuiti in modo da sostenere la pianificazione e coordinare le risorse senza compromettere l’attenzione agli aspetti operativi.
\end{itemize}
\subsection{Rotazione dei ruoli}
In seguito ai feedback rilasciati dal professore in sede di aggiudicazione appalti, il team ha discusso in merito al metodo di rotazione dei ruoli. Dopo i primi contatti con l'azienda si è deciso di ruotare i ruoli ad ogni sprint, ritenendo tale metodo quello più ragionevole, con il fine di imparare a ricoprire le "vesti" di ciascun ruolo.


\section{Preventivo dei costi}
In base al totale delle ore produttive richieste dal progetto ed alla loro suddivisione nei vari ruoli, si prevede un costo finale del progetto pari a €12.740,00.

\section{Scadenza prevista}
La scadenza stimata di consegna del prodotto terminato relativo al capitolato "BuddyBot" dell'azienda "Azzurrodigitale" è fissata per il 28 marzo 2024, considerando 20 settimane dall'aggiudicazione dell'appalto più 5 giorni cuscinetto.

\end{document}
