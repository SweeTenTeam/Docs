\documentclass[italian, 12pt]{article}
\usepackage[italian]{babel}
\usepackage[utf8]{inputenc}
\usepackage[letterpaper, left=1in, right=1in, bottom=0.75in, top=0.75in]{geometry}
\usepackage{amsmath}
\usepackage{subfiles}
\usepackage{lipsum}
\usepackage{csquotes}
\usepackage{amsfonts}
\usepackage[sfdefault]{plex-sans}
\usepackage{float}
\usepackage{pifont}
\usepackage{mathabx}
\usepackage[euler]{textgreek}
\usepackage{makecell}
\usepackage{tikz}
\usepackage{wrapfig}
\usepackage{siunitx}
\usepackage{amssymb} 
\usepackage{tabularx}
\usepackage{adjustbox}
\usepackage[document]{ragged2e}
\usepackage{floatflt}
\usepackage[hidelinks]{hyperref}
\usepackage{graphicx}
\usepackage{hyperref}
\setcounter{tocdepth}{4}
\usepackage{caption}
\usepackage{multicol}
\usepackage{tikz}
\setlength\parindent{0pt}
\captionsetup{font=footnotesize}
\usepackage{fancyhdr} 
\usepackage{graphicx}
\usepackage{capt-of}% 
\usepackage{booktabs}
\usepackage{varwidth}
\usepackage{colortbl}
\usepackage{xcolor}
\usepackage{listings}
%New colors defined below
\definecolor{codegreen}{rgb}{0,0.6,0}
\definecolor{codegray}{rgb}{0.5,0.5,0.5}
\definecolor{codepurple}{rgb}{0.58,0,0.82}
\definecolor{backcolour}{rgb}{0.95,0.95,0.92}
\definecolor{beigeCover}{rgb}{0.82, 0.77, 0.68}

%Code listing style named "mystyle"
\lstdefinestyle{mystyle}{
  backgroundcolor=\color{backcolour}, commentstyle=\color{codegreen},
  keywordstyle=\color{magenta},
  numberstyle=\tiny\color{codegray},
  stringstyle=\color{codepurple},
  basicstyle=\ttfamily\footnotesize,
  breakatwhitespace=false,         
  breaklines=true,                 
  captionpos=b,                    
  keepspaces=true,                 
  numbers=left,                    
  numbersep=5pt,                  
  showspaces=false,                
  showstringspaces=false,
  showtabs=false,                  
  tabsize=2
}

%"mystyle" code listing set
\lstset{style=mystyle}
\def\Title{Verbale Esterno}
\def\date{10/28}
% \def\VerbaleIE{} da sistemare per farlo binario
\def\Author{SweeTen Team}
\def\Redattore1{Mouad Mahdi}
\def\Verificatore1{Davide Benedetti}


\linespread{1.2}
\captionsetup[table]{labelformat=empty}

% -- TITOLO -- %
\newcommand{\customtitle}{\Title} % o ESTERNO

% -- PER LA FIRMA -- %
\newcommand{\signatureline}[1]{%
	 \par\vspace{0.5cm}
	\noindent\makebox[\linewidth][r]{\rule{0.2\textwidth}{0.5pt}\hspace{3cm}\makebox[0pt][r]{\vspace{3pt}\footnotesize #1}}%
}

% -- INTESTAZIONE -- %
\fancypagestyle{mystyle}{
	\fancyhf{} 
	\fancyhead[R]{\includegraphics[height=1cm]{images/logos/sweetenteam_nobg.png}} 
    \fancyhead[L]{\leftmark} 
   	\renewcommand{\headrulewidth}{1pt} 
  	\fancyhead[L]{\customtitle} 
	\renewcommand{\headsep}{1.3cm} 
	\fancyfoot[C]{\thepage} 
}

\begin{document}
\pagestyle{mystyle}
% Set background color for cover page
\pagecolor{beigeCover}

\include{"1 - Candidatura/lettera_di_presentazione/variables"}
\begin{titlepage}
	\begin{tikzpicture}[remember picture, overlay]
		\node[anchor=south east, opacity=0.5, yshift = -4cm, xshift= 2em] at (current page.south east) {\includegraphics[width=0.7\textwidth, trim=0cm 0cm 5cm 0cm, clip]{images/logos/Universita_Padova_transparent.png}}; 
		\node[anchor=north west, opacity=1, yshift = 8cm, xshift= 1.3cm, scale=1.6] at (current page.south west) {\includegraphics[width=4cm]{images/logos/sweetenteam.png}};
	\end{tikzpicture}
	
	\begin{minipage}[t]{0.47\textwidth}
		{\large\textbf{{\textsc{Destinatari}}}
			\vspace{3mm}
			\\ \large{\textsc{Prof. Tullio Vardanega}}
			\\ \large{\textsc{Prof. Riccardo Cardin}}
		}
	\end{minipage}
	\hfill
	\begin{minipage}[t]{0.47\textwidth}\raggedleft
		{\large\textbf{{\textsc{Redatto da: }}}
			\vspace{3mm}
			{\\\large{\textsc{ \Redattore1}\\}} % massimo due 
			
		}
		\vspace{8mm}
		
		{\large\textbf{{\textsc{Verificato da: }}}
			\vspace{3mm}
			{\\\large{\textsc{\Verificatore1}\\}} % massimo due 
			
		}
		\vspace{4mm}\vspace{4mm}
	\end{minipage}
	\vspace{4cm}
	\begin{center}
		\begin{flushright}
			{\fontsize{30pt}{52pt}\selectfont \textbf{\Title \\}}% \Date \\}} % o ESTERNO
		\end{flushright}
		\vspace{3cm}
	\end{center}
	\vspace{10 cm}
	{\small \textsc{\href{mailto: sweetenteam@gmail.com}{sweetenteam@gmail.com}}}
\pagestyle{mystyle}
\end{titlepage}

% Reset background color for the rest of the document
\pagecolor{white}


%-----------tabella versioni-----------%
\begin{table}[!h]
	\caption{Versioni}
	\begin{center}
		\begin{tabular}{ c c c p{9cm}}
			\hline \\[-2ex]
			Ver. & Data & Autore & Descrizione \\
			\\[-2ex] \hline \\[-1.5ex]

			0.1 & 28/10/2024 & Mouad Mahdi & Prima stesura del documento\\
			\\[-1.5ex] \hline
		\end{tabular}
	\end{center}
\end{table}
%---------------------------------------%


\tableofcontents
\newpage


\section{Informazioni sulla riunione}
\begin{itemize}
    \item \textbf{Luogo:} Videochiamata su Google meet
    \item \textbf{Ora di inizio:} 15:00
    \item \textbf{Ora di fine:} 15:20
    \item \textbf{Partecipanti:} Valeri Mihail Belenkov, Davide Benedetti, Matteo Campagnaro, Orlando Ferazzani, Nicolas Fracaro, Mouad Mahdi, Andrea Santi 
    \item \textbf{Partecipanti esterni:} Martina Daniele, Giorgio Vallini , Mattia Gottardello , xxxx
\end{itemize}

\section{Motivo dell'incontro}
Il team si è incontrato con i rappresentanti dell'azienda AzzurroDigitale per fare domande specifiche sul progetto e sul loro modo di lavorare con i gruppi ma anche per connoscere il loro team siccome è il progetto su cui puntiamo di più.

\section{Domande effettuate e relative risposte}
\subsection{Come deve essere l'interfaccia grafica}
I rappresentanti ci hanno specificato che abbiamo libera scelta su come fare l'intefaccia, l'unico vincolo, ovviamente, è che sia in modalità chat e che ci sia una sezione con la cronologia delle conversazioni passate fatte dall'utente(modello chatgpt).

\subsection{Collaborazioni con altri gruppi in passato}
Come per gli altri proponenti abbiamo chiesto se avessero già lavorato con altri gruppi, in particolare universitari, in passato. Ci è stato risposto di si e sempre per un progetto di ingegneria del software, uno dei rappresentanti è addirittura un ex studente del nostro corso e ci ha dato delle direttive da seguire:
\begin{itemize}
    \item Dedicare tempo alla documentazione siccome viene presa molto in considerazione
    \item Dedicare tempo all'apprendimento delle tecnologie prima di buttarsi sulla pratica
    \item Capire cosa richiederà tempo per e cosa si può fare prendendo spunto da quello che si trova in giro o da esperienze passate
    \item Non perdere tempo nel perfezionare cose prima di finire quelle che soddisfano i requisiti richiesti
\end{itemize}

\subsection{Attività che porterà via più tempo}
Abbiamo chiesto quale secondo loro è la parte del progetto più difficile. Hanno detto che la prate front-end e back-end non saranno difficili da realizzare e fattibili anche in 1-2 sprint mentre ci sarà da lavorare sulla comunicazione con servizi terzi(github, slack, ecc) e farli arrivare al servizio LLM per elaborarli.

\subsection{Modalità di comunicazione}
Volevamo sapere come avverrà la comunicazione con loro durante lo svolgimento del progetto. Ci è stato detto che una volta aggiudicato l'appalto si farà un incontro iniziale per sèecificare la way of working loro e i mezzi di comunicazione formali e anche per contatti veloci per dubbi e problemi.

\subsection{Modello AI da usare}
Abbiamo chiesto se c'è un modello specifico da usare per questo progetto. Ci hanno detto che la scelta del modello migliore per questa applicazione fa parte del lavoro di ricerca e apprendimento che dovremo fare e se c'è ne sarà bisogno ci forniranno le versioni a pagamento, con dei limiti anche quelli da definire.


\end{document}
