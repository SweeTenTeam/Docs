

\documentclass[italian, 12pt]{article}
\usepackage[italian]{babel}
\usepackage[utf8]{inputenc}
\usepackage[letterpaper, left=1in, right=1in, bottom=0.75in, top=0.75in]{geometry}
\usepackage{amsmath}
\usepackage{subfiles}
\usepackage{lipsum}
\usepackage{csquotes}
\usepackage{amsfonts}
\usepackage[sfdefault]{plex-sans}
\usepackage{float}
\usepackage{pifont}
\usepackage{mathabx}
\usepackage[euler]{textgreek}
\usepackage{makecell}
\usepackage{tikz}
\usepackage{wrapfig}
\usepackage{siunitx}
\usepackage{amssymb} 
\usepackage{tabularx}
\usepackage{adjustbox}
\usepackage[document]{ragged2e}
\usepackage{floatflt}
\usepackage[hidelinks]{hyperref}
\usepackage{graphicx}
\usepackage{hyperref}
\setcounter{tocdepth}{4}
\usepackage{caption}
\usepackage{multicol}
\usepackage{tikz}
\setlength\parindent{0pt}
\captionsetup{font=footnotesize}
\usepackage{fancyhdr} 
\usepackage{graphicx}
\usepackage{capt-of}% 
\usepackage{booktabs}
\usepackage{varwidth}
\usepackage{colortbl}
\usepackage{xcolor}
\usepackage{listings}
%New colors defined below
\definecolor{codegreen}{rgb}{0,0.6,0}
\definecolor{codegray}{rgb}{0.5,0.5,0.5}
\definecolor{codepurple}{rgb}{0.58,0,0.82}
\definecolor{backcolour}{rgb}{0.95,0.95,0.92}
\definecolor{beigeCover}{rgb}{0.82, 0.77, 0.68}

%Code listing style named "mystyle"
\lstdefinestyle{mystyle}{
  backgroundcolor=\color{backcolour}, commentstyle=\color{codegreen},
  keywordstyle=\color{magenta},
  numberstyle=\tiny\color{codegray},
  stringstyle=\color{codepurple},
  basicstyle=\ttfamily\footnotesize,
  breakatwhitespace=false,         
  breaklines=true,                 
  captionpos=b,                    
  keepspaces=true,                 
  numbers=left,                    
  numbersep=5pt,                  
  showspaces=false,                
  showstringspaces=false,
  showtabs=false,                  
  tabsize=2
}

%"mystyle" code listing set
\lstset{style=mystyle}
\def\Title{Verbale Esterno 18/10/2024}
% \def\VerbaleIE{} da sistemare per farlo binario
\def\Author{SweeTen Team}
\def\Redattore1{Nicolas Fracaro}
\def\Verificatore1{Orlando Ferazzani}


\linespread{1.2}
\captionsetup[table]{labelformat=empty}

% -- TITOLO -- %
\newcommand{\customtitle}{\Title} % o ESTERNO

% -- PER LA FIRMA -- %
\newcommand{\signatureline}[1]{%
	 \par\vspace{0.5cm}
	\noindent\makebox[\linewidth][r]{\rule{0.2\textwidth}{0.5pt}\hspace{3cm}\makebox[0pt][r]{\vspace{3pt}\footnotesize #1}}%
}

% -- INTESTAZIONE -- %
\fancypagestyle{mystyle}{
	\fancyhf{} 
	\fancyhead[R]{\includegraphics[height=1cm]{images/logos/sweetenteam_nobg.png}} 
    \fancyhead[L]{\leftmark} 
   	\renewcommand{\headrulewidth}{1pt} 
  	\fancyhead[L]{\customtitle} 
	\renewcommand{\headsep}{1.3cm} 
	\fancyfoot[C]{\thepage} 
}

\begin{document}
\pagestyle{mystyle}
% Set background color for cover page
\pagecolor{beigeCover}

\include{"1 - Candidatura/lettera_di_presentazione/variables"}
\begin{titlepage}
	\begin{tikzpicture}[remember picture, overlay]
		\node[anchor=south east, opacity=0.5, yshift = -4cm, xshift= 2em] at (current page.south east) {\includegraphics[width=0.7\textwidth, trim=0cm 0cm 5cm 0cm, clip]{images/logos/Universita_Padova_transparent.png}}; 
		\node[anchor=north west, opacity=1, yshift = 8cm, xshift= 1.3cm, scale=1.6] at (current page.south west) {\includegraphics[width=4cm]{images/logos/sweetenteam.png}};
	\end{tikzpicture}
	
	\begin{minipage}[t]{0.47\textwidth}
		{\large\textbf{{\textsc{Destinatari}}}
			\vspace{3mm}
			\\ \large{\textsc{Prof. Tullio Vardanega}}
			\\ \large{\textsc{Prof. Riccardo Cardin}}
		}
	\end{minipage}
	\hfill
	\begin{minipage}[t]{0.47\textwidth}\raggedleft
		{\large\textbf{{\textsc{Redatto da: }}}
			\vspace{3mm}
			{\\\large{\textsc{ \Redattore1}\\}} % massimo due 
			
		}
		\vspace{8mm}
		
		{\large\textbf{{\textsc{Verificato da: }}}
			\vspace{3mm}
			{\\\large{\textsc{\Verificatore1}\\}} % massimo due 
			
		}
		\vspace{4mm}\vspace{4mm}
	\end{minipage}
	\vspace{4cm}
	\begin{center}
		\begin{flushright}
			{\fontsize{30pt}{52pt}\selectfont \textbf{\Title \\}}% \Date \\}} % o ESTERNO
		\end{flushright}
		\vspace{3cm}
	\end{center}
	\vspace{10 cm}
	{\small \textsc{\href{mailto: sweetenteam@gmail.com}{sweetenteam@gmail.com}}}
\pagestyle{mystyle}
\end{titlepage}

% Reset background color for the rest of the document
\pagecolor{white}


%-----------tabella versioni-----------%
\begin{table}[!h]
	\caption{Versioni}
	\begin{center}
		\begin{tabular}{ c c c p{9cm}}
			\hline \\[-2ex]
			Ver. & Data & Autore & Descrizione \\
			\\[-2ex] \hline \\[-1.5ex]
			0.1 & 23/10/2024 & Nicolas Fracaro& Prima stesura del documento\\
			\\[-1.5ex] \hline
		\end{tabular}
	\end{center}
\end{table}
%---------------------------------------%


\tableofcontents
\newpage

\section{Contenuti del verbale}

\subsection{Informazioni sull'incontro}
\begin{itemize}
    \item \textbf{Luogo:} Scambio di email all'indirizzo fornito nella presentazione del capitolato di \textit{AzzurroDigitale}
    \item \textbf{Partecipanti:} Valeri Mihail Belenkov, Davide Benedetti, Matteo Campagnaro, Orlando Ferazzani, Nicolas Fracaro, Mouad Mahdi, Andrea Santi
    \item \textbf{Partecipanti esterni:} Nicola Boscaro (rappresentante di \textit{AzzurroDigitale})
\end{itemize}

\section{Obiettivi dell'incontro}
Durante l'incontro sono stati trattati i seguenti temi:
\begin{itemize}
    \item Chiarire gli obiettivi e le tecnologie da utilizzare per il progetto presentato nel capitolato 9.
\end{itemize}

\section{Sintesi dell'incontro}
Il team ha concordato di inviare direttamente domande specifiche via email per velocizzare le risposte, riducendo al minimo il tempo richiesto all'azienda proponente.

\section{Domande effettuate e relative risposte}
\subsection{BuddyBot deve accedere solo alle repository interne di \textit{AzzurroDigitale} o anche a repository pubbliche?}
L'azienda ha confermato che è richiesto l'accesso esclusivo alle repository private tramite API.

\subsection{Come viene gestita l'autenticazione e l'accesso alle diverse fonti di \textit{AzzurroDigitale} da parte dei vari utenti di BuddyBot?}
L'azienda ha indicato che, per gli scopi di questo progetto, non è necessario implementare una gestione capillare dei permessi utente, confermando, però, che potrebbe essere uno sviluppo futuro interessante. L'integrazione con le fonti (Confluence, Jira, GitHub) avverrà su spazi di progetto specifici.

\subsection{Cosa si intende per "sistema di Bug Reporting"?}
L'azienda ha indicato l'uso della sezione "Issues" delle repository GitHub come metodo per fornire feedback su eventuali bug riscontrati.

\subsection{È consentito l'uso di tecnologie diverse da quelle consigliate da \textit{AzzurroDigitale} nel capitolato?}
L'azienda ha confermato che le tecnologie consigliate sono opzionali, lasciando libertà di utilizzare tecnologie alternative, come ad esempio React invece di Angular.



\end{document}